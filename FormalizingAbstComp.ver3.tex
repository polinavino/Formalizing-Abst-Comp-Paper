\documentclass{entcs} \usepackage{entcsmacro}
\usepackage{graphicx}
%\sloppy
% The following is enclosed to allow easy detection of differences in
% ascii coding.
% Upper-case    A B C D E F G H I J K L M N O P Q R S T U V W X Y Z
% Lower-case    a b c d e f g h i j k l m n o p q r s t u v w x y z
% Digits        0 1 2 3 4 5 6 7 8 9
% Exclamation   !           Double quote "          Hash (number) #
% Dollar        $           Percent      %          Ampersand     &
% Acute accent  '           Left paren   (          Right paren   )
% Asterisk      *           Plus         +          Comma         ,
% Minus         -           Point        .          Solidus       /
% Colon         :           Semicolon    ;          Less than     <
% Equals        =3D           Greater than >          Question mark ?
% At            @           Left bracket [          Backslash     \
% Right bracket ]           Circumflex   ^          Underscore    _
% Grave accent  `           Left brace   {          Vertical bar  |
% Right brace   }           Tilde        ~

% A couple of exemplary definitions:

\newcommand{\Nat}{{\mathbb N}}
\newcommand{\Real}{{\mathbb R}}
\def\lastname{Vinogradova, Felty, and Scott}
\begin{document}
	\begin{frontmatter}
%\title{A Framework for Studying Category\\ Theory and Computability}
%\title{A Logical Framework for Studying Category Theory and Computability} 
\title{Formalizing Abstract Computability:\\ Turing Categories in Coq}
		\author[CS]{Polina Vinogradova\thanksref{emailpv}}
		\author[CS,Math]{Amy P. Felty\thanksref{emailaf}}
		\author[Math,CS]{Philip Scott\thanksref{emailps}}
		\address[CS]{School of Electrical Engineering and Computer Science, University of Ottawa, Ottawa, Canada}
		\address[Math]{Department of Mathematics and Statistics, University of Ottawa, Ottawa, Canada}
		\thanks[emailpv]{Email: polina.vino@gmail.com}
		\thanks[emailaf]{Email: afelty@uottawa.ca}
		\thanks[emailps]{Email: phil@site.uottawa.ca}


\begin{abstract} 
  The concept of computable functions (as developed by G\"odel, Church,
  Turing, and Kleene in the 1930's) has been extensively studied,
  leading to the modern subject of recursive function theory. However
  % with the development of category-theoretic methods
  recent work by category theorists has led to a more conceptual and
  abstract foundation of computability theory--Turing categories.
  % it has become possible to study recursion in a more abstract
  % sense.  The particular model this paper is structured around is
  % known as a Turing category.  The structure within
  A Turing category models the notion of partial map as well as
  recursive computation, using methods of categorical algebra to
  formalize these concepts.
  % The goal of this work is to build a formal language description of
  % this categorical model of computation. Specifically, we use the
  % Coq proof assistant to formulate informal definitions,
  % propositions and proofs pertaining to Turing categories.  This is
  % developed in the underlying formal language of Coq, the Calculus
  % of Co-inductive Constructions (CIC). Furthermore, we instantiate
  % the more general Turing category formalism with a CIC description
  % of the category which explicitly models the language of partial
  % recursive functions.
  The goal of this work is to provide a formal framework for analyzing
  this categorical model of computation.  We build on an existing
  library for general category theory developed in the Coq Proof
  Assistant.  We focus on both formalizing Turing categories and on
  building a general framework in the form of a well-structured Coq
  library that can be further extended.
  % that can be adopted by others to study further properties in this domain.
  We begin by formulating informal definitions, propositions, and
  proofs pertaining to Turing categories, and then instantiate the
  more general Turing category formalism with a CIC description of the
  category which explicitly models the language of partial recursive
  functions.
\end{abstract}
\begin{keyword}
Category theory, Turing categories, Computability, Formalization, Calculus of Inductive Constructions (CIC), Coq proof assistant
\end{keyword}		
\end{frontmatter}




\section{Introduction}\label{intro}

Traditional computation theory (G\"odel, Turing, Kleene) originally aimed at capturing the 
informal notion of computable functions over the natural numbers and computable theories of real numbers
(e.g. \cite{Turing-Paper,IMM}).  Even before Turing, Church's introduction of lambda calculus \cite{Church}  attempted to capture certain properties of computation in a broader sense,
through manipulation of strings of symbols representing function application and abstraction. It was subsequently shown that the
numerical (partial) functions computable by these various abstract formalisms for computation coincided, thus suggesting the so-called Church-Turing thesis.  Later independent formalisms
for defining numerical computation (e.g. Register Machines, Markov's algorithms, etc.) were shown to again lead to the same class of computable numerical partial functions, lending yet more credance to the Church-Turing thesis.  Nevertheless, practical as well as theoretical computer science requires more than just numerical computation:  one has increasingly abstract theories of computation over various data types, of higher-order computation, computation based on various programming language paradigms, newer paradigms of computation (parallel, probabilistic, quantum) etc.  Category theory appears to be both general and expressive enough to be 
the tool of choice for modeling computation in a these newer senses. 

Many concepts of traditional recursion and computation theory have begun to be effectively analyzed categorically.  This includes early fundamental work of Elgot on flowchart semantics,
and its connections with denotational semantics  (cf.  M. Arbib and E. Manes \cite{MA}), to 
analysis of Church's theories of lambda calculi (both typed and untyped) in cartesian closed categories and associated higher-order categorical logics (\cite{HOCL}).  In a series of works
of increasing categorical generality, beginning with Longo and Moggi \cite{LMog}, Di Paola and Heller \cite{DiPao} and culminating in recent work of Cockett and Hofstra \cite{Turing}, we see the beginnings of a new and direct categorical development of the foundations of recursion theory 


This paper is based on the thesis of the first author~\cite{MyPhD}.
The model we study is the category-theoretic formalism of ``Turing Categories", introduced by Cockett and Hofstra in~\cite{Turing}. Turing categories are a very general computational model, built from a categorical analysis of 
partial maps in categories. The partial maps of a Turing category arise as the computable maps of a partial combinatory algebra (PCA), see \cite{Turing}.  Moreover,  recent work establishes criteria for 
determining when various complexity classes of total maps  can be made into a Turing category
\cite{Total}.  Thus the notion of Turing category provides a robust, abstract framework for discussing computation over a wide range of settings. 

Our study of the Turing category computation model takes the form of building a type-theoretic formal language description (formalization) of the relevant concepts. The concepts we have selected to formalize lay the groundwork for (formally) proving abstract interpretations of standard theorems in recursion theory. The key motivation behind  this approach is the level of organization, consistency, and guaranteed correctness it provides in working with proofs and definitions for which informal formulations may omit important and interesting details. 

%Note that there has been another project formalizing restriction structure using Agda

Turing category theory can be viewed as an (up until now) informally-presented mathematical framework that can be used to describe formal computation. As computation on a physical computer is a precise procedure, it seems natural to verify that a formal description of it exactly fits the selected categorical model. This is the motivating idea and the main objective of this work. There is not a huge amount of work done in this direction of research; specifically, in formalizing a category as an instance of an abstract computational model. Furthermore, because of the nature of Coq, we are using intuitionistic logic to build the proofs and definitions in this formalization.  This further differentiates this development from traditional recursion theory, and adds interesting constructivist information to our proofs. For example, to verify if $f: A\rightarrow B$ is a function, we must confirm that for each proof that $x\in A$, we can prove $f(x)\in B$.

%Our intent was to first use an existing formal language and category theory library written in that language to specify the mathematical definitions found in the framework of the Turing Category abstract computation model, as well as the abstracted versions of other types of structures naturally occurring in the traditional computation model, then formally prove (the more abstract versions of) a number of the results from traditional recursion theory. The proof assistant chosen for this purpose is Coq, with the Calculus of (co)Inductive constructions as the underlying formal language, and . The existing category theory library we chose on top of which to build the Turing category formalization and examples, described in~\cite{timanyNewLib}. We also chose to adopt the style of definitions and formalization strategy used in this library. It appears to be the most comprehensive, easy to use and well-structured category theory Coq library. 

The proof assistant chosen for this purpose is Coq, with the Calculus
of (co)\-In\-duc\-tive Constructions (CIC) as the underlying formal
language.  We start from a library for general category theory
developed by Timany and Jacobs~\cite{timanyNewLib}, designed to take
advantage of advanced features in Coq 8.5 such as universe
polymorphism.  This library successfully develops many of the basic
concepts, and thus we chose to adopt the style of definitions and
formalization strategy used in this library.  With this library as a
starting point, we specify the mathematical definitions found in the
framework of the Turing Category computation model, as well
as abstract versions of other types of structures naturally
occurring in the traditional computation model.  We  then formally
prove (the abstract versions of) a number of results from
traditional recursion theory.

%In addition to formalizing the categorical concepts, the next key part of this project is formalizing several examples of categories which contain the types of structure relevant to the theory of Turing categories. These examples serve as test cases for the verification of the validity of the formal versions of the definitions and propositions constructed in the formal language (and hence, the related formally proved results about them) as well as to formally study these categories. The main example we focus on in this project is the formalization of traditional computation on the natural numbers and the categorical interpretation of all the structure found therein in order to prove that these indeed conform to the Turing category model formalism. We base our formalization of traditional recursion theory on a formalization due to V. Zammit~\cite{SmnForm}. 

In addition to formalizing the categorical concepts, we formalize
several examples of categories. These examples provide validation
of our formalization approach and formalized results.  They also
provide a mechanism to formally study these specific example
categories. Our main example is the formalization of traditional
computation on the natural numbers and the categorical interpretation
of all the structure found therein, illustrating that these indeed
conform to the Turing category model formalism. We base our
formalization of traditional recursion theory on a formalization due
to Zammit~\cite{SmnForm}.

Our Coq scripts, compilation instructions, and a link to the library
we build on is available at:
\href{https://github.com/polinavino/Turing-Category-Formalization}
{\texttt{https://github.com/polinavino/Turing-Category-Formalization}}.

\section{Our Formalization}

%In this paper, we divide our explanation of the formalization project according to the Coq code files we have built, each of which contains the code for a particular type of structure or example. The section corresponding to each file discusses the informal definitions which the code formalizes, the challenges of reconciling the differences between formal and informal definitions, and existing related work. For the complete code, compilation instructions, and link to the original library over top of which our formalization is built, see
%
%\href{https://github.com/polinavino/Turing-Category-Formalization}
%{\texttt{https://github.com/polinavino/Turing-Category-Formalization}}

In this section, we divide our explanation of the formalization
according to the Coq code files we have built. The section
corresponding to each file discusses the informal definitions which
the code formalizes, the challenges of reconciling the differences
between formal and informal definitions, and existing related work.

{\bfseries Restriction.} In order to model partial computation using category theory, we need to first axiomatize partiality in a category. In a Turing category, this is done in terms of a restriction combinator~\cite{Restriction}. A restriction combinator axiomatizes partial maps in terms of idempotents. In addition to a partiality structure, a category wherein we are able to axiomatize computation requires a version of cartesian product structure which interacts well with the restriction structure. The partial versions of products and terminal objects are called restriction (or partial) products and restriction-terminal objects, respectively. A category which admits these structures is called a {\em cartesian restriction category.} 

%takes a map $f : A \to B$ in a category $\mathsf{C}$ to an idempotent $\overline{f} : A \to A$ in a way that satisfies certain rules, and  in a sense axiomatizes the notion of `domain' of $f$. When a category $\mathsf{C}$ admits such a combinator, it is called a {\em restriction category}. 
%Now, requiring strict cartesian structure to exist in a restriction category does not provide us with the correct machinery to perform computation. Thus, a weaker version of cartesian structure, which interacts well with restriction structure has been defined~\cite{Turing}. 

In order to formalize these concepts we have followed the style of the Coq category theory library we have selected, using type classes to formalize categorical notions. Type classes are a versatile and convenient way to encapsulate terms and propositions about them into a single term representing its informal counterpart, with a number of features particularly useful for reasoning about category theory. We have also formalized and proved a number of results about cartesian restriction structure, including: 

\begin{enumerate}
	\item A restriction terminal object in a cartesian restriction category is a (true) terminal object in its total subcategory, and a corresponding result about restriction products~\cite{Turing};
	
	\item Restriction products in a cartesian restriction category are (true) products in its total subcategory~\cite{Turing};
	
	\item A cartesian category with $\overline{f} = 1$ for all maps $f$ is a cartesian restriction category~\cite{Turing};
	
	\item In an embedding-retraction pair $(m, r)$, $m$ is a total map~\cite{Restriction};
	
	\item A ranges and retractions lemma from~\cite{MyThesis}.
\end{enumerate}

%For example, the type class representing the notion of a restriction combinator takes a parameter {\tt C : Category}, which will be the category in which the restriction combinator is being defined. In the body of the type class, there is a term {\tt rc} which, given {\tt a b : C}, and a map {\tt f : Hom a b} returns a map of type {\tt Hom a a}. In addition, in the body of the class, there are terms {\tt rc1, rc2, ...}, the types of which correspond the axioms {\tt rc} must satisfy. The cartesian restriction structure is formalized using a similar approach, as are notions of a total subcategory, trivial restriction structure, etc~\cite{Turing}. 


{\bfseries PCA and CompA.} The underlying computational structure in a Turing category is a non-associative algebra called a partial combinatory algebra (abbreviated PCA), which consists of a pair $(A, \bullet)$ of an object $A \in \mathsf{C}$ and a map $A \times A \to A$ in a cartesian restriction category $\mathsf{C}$, as well a combinatory completeness condition that $(A, \bullet)$ must satisfy in this category. Given an object $A$ in an arbitrary restriction category $\mathsf{C}$, the full subcategory of all objects of the form $A^n$ is denoted Comp($A$). We have formalized the definition of PCA, as well as Comp($A$) and Split(Comp($A$)) (the Karoubi envelope of Comp($A$)) as cartesian restriction categories. Furthermore, in the process, we have discovered additional conditions required to show intuitionistically that Split(Comp($A$)) is a cartesian restriction category. 


%, or Comp($\mathbb{A}$), whenever a partial combinatory algebra structure $(A, \bullet)$ exists in $\mathsf{C}$ (and is therefore in Comp($\mathbb{A}$) also). The Karoubi envelope, i.e., the category of formally split idempotents, of the category Comp($A$) is denoted Split(Comp($A$)), or Split(Comp($\mathbb{A}$)), whenever $\mathsf{C}$ admits a PCA. 

{\bfseries Turing.} A Turing category is a category that contains a special kind of structure that models computation. A category $\mathsf{T}$ is Turing if it contains an object $A \in \mathsf{T}$, called a Turing object, as well as a morphism $\bullet : A \times A \to A$, called a Turing morphism, such that each map in $\mathsf{T}$ factors via $\bullet$ in a specific way, similar to the factorization of maps within cartesian closed categories~\cite{HOCL}. We have formalized Turing structure along with a number of results about it, including:

 %Each object in $\mathsf{T}$ is a retract of the Turing object. Now the pair $(A, \bullet)$ is called a Turing structure and constitutes a PCA. This structure is not necessarily unique in $\mathsf{T}$.

\begin{enumerate}
\item Every object in a Turing category is a retract of a Turing object~\cite{Turing};
\item An object $B$ in a Turing category with Turing object $A$ is Turing if and only if it is a retract of $A$~\cite{Turing};
	
\item A CCC with trivial restriction structure and an object $A$ of which every object is a retract is a Turing category;
\item The halting domain is $m$-complete~\cite{Turing};
	
\item An equivalent characterization of Turing categories in terms of the Turing morphism and object embeddings~\cite{Turing}.
	
\end{enumerate}

In addition, we have formalized the relationship between a Turing category $\mathsf{T}$ with a Turing object $A$ and the related categories Comp($A$) and Split(Comp($A$)). These categories embed in the following order:
Comp($\mathbb{A}$) $\subseteq$ $\mathsf{T}$ $\subseteq$ Split(Comp($\mathbb{A}$)).

{\bfseries Range.} Range structure in a category can be expressed in terms of another type of combinator which (whenever it exists in a category) is in a sense dual to the restriction combinator~\cite{RangeI}. The range combinator takes a map $f : A \to B$ to a map $\hat{f} : B \to B$, and, as with the restriction combinator, satisfies a number of axioms. It is related to the notion of open maps in a category, in that whenever all maps in a given category are open, it is a range category. We have chosen to formalize this particular abstraction because in the process of formalizing the motivating examples, %of the categories of sets and partial maps $\mathsf{Par}$ and the category of computable maps between %$n$-fold products of the natural numbers, Comp($\mathbb{N}$), 
it became apparent that representing partiality using a total formal language presented one of the biggest challenges as well as one of the greatest curiosities. We have formalized a number of results regarding the interactions between range structure and embedding-retraction pairs, as well as a criterion for a Turing category to admit cartesian range structure~\cite{MyThesis}.

{\bfseries Par\_Cat.} The category $\mathsf{Par}$ is the motivating example for the categorical structure discussed above, including cartesian restriction and cartesian range structure, but not including Turing or PCA structure. Due to the total nature of computation in the CIC, it is impossible to directly represent a partial map. For this reason, we must define the type of the set of all maps $A \to B$ quite differently from the type of all total maps $A \to B$ in the category $\mathsf{Set}$ (i.e., {\tt A -> B}). This set of partial maps is a pair consisting of a domain predicate {\tt P : A -> Prop} together with a map of type {\tt forall x:A, P x -> B}, which takes two arguments: an `element' {\tt x} of the set {\tt A}, and a proof of the proposition {\tt P x}. 

Having defined the type of partial maps in this way, we proceeded to instantiating the {\tt Category} type class with the required objects, morphisms and proofs of associativity, etc. Following a similar format, we have also instantiated $\mathsf{Par}$ as a cartesian restriction category and a cartesian range category (i.e., defined all the required maps, objects, and completed the accompanying proofs).

%Using this category, we can now instantiate the restriction combinator. 
%The restriction of a given map {\tt f : Hom a b} has the same domain predicate {\tt P} as the map {\tt f} itself, but the restriction of {\tt f} evaluated at {\tt x : A} and a proof {\tt pf : P x}. Note that once we have defined the restriction combinator mapping, we must also prove that it satisfies the required axioms in order to complete the instantiation. 

%More specifically, we are not using the language to represent a specific (explicitly defined) map, but rather we must be able compare two arbitrary maps, presenting a proof of an equality judgment. 
{\bfseries CompN\_Cat.} The category $\mathsf{Par}$ contains a subcategory of maps that are partial recursive, i.e., computable by a map which can be expressed in terms of the partial recursive constructors (zero, successor, projection, recursion, substitution and minimalization~\cite{Computability}). We use this (informally-presented) definition of formal computation as the basis for our formalization of the category of computable maps. We build our subcategory using an existing fomalization of this presentation of computation as well as the proof of the $S^m_n$ theorem completed using this definition~\cite{SmnForm}.

This formalization gives the definition and the semantics of the
language of partial recursive maps separately. We define the language
constructors as an inductive type {\tt prf} in Coq, while the
semantics are given as an inductively-defined relation, whose header is:

{\tt \small Inductive {\bfseries converges\_to} : prf -> list nat -> nat -> Prop} 

\noindent where {\tt convergest\_to f\_prf ln n} is provable whenever,
informally, the partial recursive function {\tt f\_prf} applied to the
list of natural numbers {\tt ln} outputs {\tt n}. Note that this
`output' is unique, so we are able to build a partial map in the {\tt
  Par\_Cat} sense, described above, which corresponds to a given {\tt prf} term. 
%We use this fact to build a category 
%In order to build a map that is Kleene-equal to the computation a particular {\tt f\_prf : prf} represents, we use the {\tt converges\_to} relation to give the domain predicate, and add a special case of an axiom of choice to select, given a list {\tt ln: list nat}, a natural number as output \textit{in   the case when there is a proof that this output necessarily exists} (i.e., a proof {\tt pf : exists n, converges\_to f\_prf ln n}).

Now, we formally consider a partial map with $m$ components as a map in the category of partial recursive maps whenever there is a proof that each of its components is Kleene-equal to a {\tt prf} computation. This category inherits cartesian restriction structure defined in the larger category, {\tt Par\_Cat}; however, we have additionally defined a Turing object (i.e., the natural numbers) and proved that the necessary diagrams commute (as maps in the larger {\tt Par\_Cat} category), thus demonstrating that it is indeed a Turing category.

\section{Conclusion and Future Work}

The framework we have built develops the  tools to study
abstract recursion formally. It has the advantage that we are not
required to model partial functions using total functions or relations
(or total functions built using relations) in order to study partial
recursion using (strongly normalizing and intuitionistic) CIC.

There are a number of promising directions
%to advancing this formalization
for further applying this framework.  The most natural, perhaps, is
the formalization of the Turing category-formulated abstraction of
Rice's theorem.  This will require the formalization of a number of
general categorical concepts such joins and meets, as emphasized in
Cockett's lectures ~\cite{Estonia}.  Such general concepts are widely
applicable to other results.  Applying our framework in another
direction, it would be an extremely interesting and innovative pursuit
to use it to study computational complexity classes of total maps in
Turing categories. Other potentially interesting options for building
on this framework include formalizing monoidal Turing categories (with
differential structure) and conducting a formal study more focused on
the PCA's (which, recall, are computation-modeling structures at the
core of every Turing category) as well as relationships between them.



%\begin{thebibliography}\label{thesisbib}

\bibliographystyle{entcs}
\bibliography{thesisbib}


%\end{thebibliography}

\end{document}